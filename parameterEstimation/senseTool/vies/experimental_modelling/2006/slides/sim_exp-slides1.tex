\documentclass[pdf,slideColor]{prosper}

% \ifinColor
%         \newcommand{\mathColor}{\green}
%         \newcommand{\emphColor}{\red}
%         \newcommand{\itemColor}{\red}
%         \newcommand{\titleColor}{} % Automatic ...
%         \newcommand{\pslineColor}{black}
%         \newcommand{\psemphlineColor}{red}
% \else
%         \newcommand{\mathColor}{}
%         \newcommand{\emphColor}{\it}
%         \newcommand{\itemColor}{}
%         \newcommand{\titleColor}{\black} % Override ...
%         \newcommand{\pslineColor}{black}
%         \newcommand{\psemphlineColor}{black}
% \fi
%% \FontTitle{%
%%   \usefont{T1}{ptm}{b}{n}\fontsize{20.74pt}{20pt}\selectfont%
%%   \lightgray}{%
%%   \usefont{T1}{ptm}{b}{n}\fontsize{20.74pt}{20pt}\selectfont%
%%   \black}
%% \newcommand{\mf}[1]{{\mathColor$#1$}}
%% \newcommand{\dmf}[1]{{\mathColor$$#1$$}}
%% \newcommand{\reel}{\mathcal{R}}
%% \newcommand{\Z}{\mathcal{Z}}
%% \newcommand{\cemph}[1]{\emph{\emphColor #1}\xspace}
%% \newcommand{\cbul}{\item[\mbox{\red{$\bullet$}}]}
%% \newcommand{\devDepStuff}{}%
%% \usepackage{xspace}

%\newlength{\jssignalwidth}

\usepackage[latin1]{inputenc}
%\usepackage{pstricks,pst-node,pst-text,pst-3d}
\usepackage{graphicx,theorem}
\usepackage{amsmath,amsfonts,amssymb}
\graphicspath{{figurer/}}
\renewcommand{\vec}[1]{\boldsymbol{#1}}
\newcommand{\argmin}{\mathop{\mathrm{argmin}}}
\theoremstyle{break}\newtheorem{Saet}{}

\title{\bigskip
Simulering og eksperimentel modelbestemmelse}
%\subtitle{ \ }
\author{\href{http://www.control.aau.dk/~vie/}{Henrik Vie
      Christensen}}
\institution{%
Department of Control Engineering\\
Aalborg University\\
Denmark}
\slideCaption{Sim.exp.model.1}

\email{vie@control.aau.dk}

%\DefaultTransition{Wipe}

\begin{document}
\maketitle

%---------------------------------------------------------------------- SLIDE -
\overlays{5}{%
\begin{slide}[Replace]{Kursusoversigt}

\bigskip
Plan for de enkelte minimoduler:

\bigskip
\begin{itemstep}
\item[1.] Introduktion, metode og procedure for eksperimentel
modelbestemmelse, Grafisk modeltilpasning, System identifikation
\item[2.] Modellering, modelbeskrivelse og simulering
\item[3.] Senstools til parameterestimering
\item[4.] Parameter n�jagtighed og f�lsomhed, Frekvensdom�net
\item[5.] Design af inputsignaler
\end{itemstep}

\end{slide}
}
%---------------------------------------------------------------------- SLIDE -
\overlays{7}{
\begin{slide}[Replace]{\large Eksperimentel modelbestemmelse, Senstools}

Metodens fordele:
\begin{itemstep}
\item En simpel grundl�ggende metode, illustreret grafisk.
\item Modeller i kontinuert tid med fysisk betydende parametre.
\item Enhver model struktur kan anvendes, line�r, uline�r,
  fordelte parametre, tidsforsinkelse, etc.
\item Stokastiskeaspekter er reduceret til et minimum.
\item Robust over for afvigelser fra teoretiske antagelser.
\item Et f�lsomheds metode brugbar til valg af model struktur,
  eksperiment design, og n�jagtighedsverifikation.
\item Alt i alt, kompatibel med fysisk indsigt.
\end{itemstep}
\hspace*{\fill} Morten Knudsen

\end{slide}
}
\addtocounter{equation}{2}
%---------------------------------------------------------------------- SLIDE -

\begin{slide}[Replace]{Applikationer}

Senstools og f�lsomhedsmetoden for eksperimentel modellering er blevet
anvendt i mange forsknings og studenter projekter. Eksempler er:
\begin{itemize}
\item Skibs- og maritime systemer
\item Vindm�ller
\item H�jtalere
\item Induktions- og DC-motore
\item Varmevekslere
\item Menneskeligt v�v for hypertermi-terapi mod kr�ft
\item Nyre og cerebellar blodgennemstr�mning
\end{itemize}

\end{slide}

%---------------------------------------------------------------------- SLIDE -
\overlays{5}{
\begin{slide}[Replace]{Procedure for eksperimentel modellering}

\vspace*{-4mm}
\hspace*{\fill}\includegraphics[width=.85\linewidth]{sys_ident_principle.md}\hspace*{\fill}
%
%\medskip
\vspace*{-2mm}
\onlySlide*{1}{
\begin{description}
\item[\normalsize\blue Bestemmelse af modelstruktur:] Modelstrukturen er
  bestemt af basal fysisk indsigt og empiriske overvejelser. En
  ``{\blue simuleringsmodel}'' konstrueres.
\end{description}
}%
\onlySlide*{2}{
\begin{description}
\item[\normalsize\blue Eksperiment design:] Specielt vigtigt er et ``{\blue
    godt}'' {\blue inputsignal}. 
%Det er enten valgt ud fra fornufts
%    overvejelser, eller designet som et signal der optimerer
%    karakteristiske parameter f�lsomhedsm�l.
\end{description}
}%
\onlySlide*{3}{
\begin{description}
\item[\normalsize\blue Eksperiment:] {\blue Systemet exciteres med inputsignalet} og
  overensstemmende v�rdier af input- og outputsignaler samples og
  gemmes.
\end{description}
}%
\onlySlide*{4}{
\begin{description}
\item[\normalsize\blue Parameter estimation:] Simulationsmodellens {\blue parametre
  justeres} til minimum afvigelse mellem det samplede system output og
  modellen.% output (model fejlen). Som et m�l for
%  afvigelsen bruges en performancefunktion.
\end{description}
}
\onlySlide*{5}{
\begin{description}
\item[\normalsize\blue Model validering:] Korrektheden af {\blue modelstrukturen} og
  {\blue n�jagtigheden af parameter estimaterne kontrolleres}.%, baseret p�
% ``{\blue model fit}'' og karakteristisk parameter f�lsomhedsm�l.
\end{description}
}
\end{slide}
}
%---------------------------------------------------------------------- SLIDE -
\overlays{4}{
\begin{slide}[Replace]{Eksempel 1: Grafisk model tilpasning}
Bestem forst�rkning $\blue K$ og tidskonstant $\blue\tau$ ved at tilpasse en
f�rste-ordens model til den m�lte step-respons:

\untilSlide*{3}{
\begin{minipage}[c]{.6\linewidth}
\includegraphics[width=\linewidth]{step_response_measured.md}
\end{minipage}%
}%
\fromSlide*{4}{
\begin{minipage}[c]{.6\linewidth}
\includegraphics[width=\linewidth]{step_response.md}
\end{minipage}%
}%
\onlySlide*{2}{%
\begin{minipage}[c]{.39\linewidth}
Model:

$\blue G_m(s) = \frac{K}{1+s\tau}$

\medskip
Input: (step)

$\blue U(s) = \frac{a}{s}$

\medskip
Step respons:

$\blue Y(s) = \frac{aK}{s(1+s\tau)}$
\end{minipage}
}%
\fromSlide{3}{
\begin{minipage}[c]{.39\linewidth}
I tidsdom�net:

$\blue y(t) = aK(1 - e^{-\frac{t}{\tau}})$

\bigskip
$t \rightarrow \infty:$\\
$y(\infty) = aK$\\
$\Rightarrow \blue K = \frac{y(\infty)}{a}$

\medskip
$t = \tau:$\\
${\blue y(\tau)} = aK(1 - e^{-1})$\\
\hspace*{25.7pt}$\blue = 0.63 aK$
\end{minipage}
}%

\end{slide}
}
%---------------------------------------------------------------------- SLIDE -

\begin{slide}[Replace]{Eksempel 2: Grafisk model tilpasning}

Tilsvarende for et f�rsteordens-system med forsinkelse $\blue T$:

\medskip
\begin{minipage}[c]{.6\linewidth}
\includegraphics[width=\linewidth]{delayed_step_response.md}
\end{minipage}%
\begin{minipage}[c]{.39\linewidth}
Model:

$\blue G_m(s) = \frac{K}{1+s\tau} e^{-sT}$
\end{minipage}

\end{slide}

%---------------------------------------------------------------------- SLIDE -
\overlays{3}{
\begin{slide}[Replace]{Systemidentifikationsmetoder}

Metoderne er karakteriseret af modeltyperne:

\begin{itemstep}
\item \textbf{Linear diskrete-time model:} \quad Klassisk
  systemidentifikation
\item \textbf{Neuraltnetv�rk:} \quad Meget uline�re systemer med en
  kompliceret struktur
\item \textbf{Generel simulationsmodel:} \quad Enhver matematisk
  model, som kan simuleres fx. med Matlab. Den kr�ver en {\blue
  fysisk realistisk model struktur}, typisk udviklet ved teoretisk
  modellering.\\
Metoden: Direkte estimering af fysiske parametre
\end{itemstep}

\end{slide}
}
%---------------------------------------------------------------------- SLIDE -

\begin{slide}[Replace]{Computer tilpasning ved minimering}

\begin{minipage}[c]{.39\linewidth}
Performance funktion:
$$\blue P(\theta) = \frac{1}{2N} \sum_{k=1}^N \varepsilon^2(k,\theta)$$

\medskip
Optimale parametre:
$$\blue\theta_N = \argmin_\theta P(u_N, y_N, \theta)$$
\end{minipage}%
\begin{minipage}[c]{.6\linewidth}
\includegraphics[width=\linewidth]{model_error_kT.md}
\end{minipage}%

hvor $T$ er samplingstiden og $\varepsilon(k,\theta) = y(kT) - y_m(kT,\theta)$.

\end{slide}

%---------------------------------------------------------------------- SLIDE -

\begin{slide}[Replace]{Performance funktion som fkt. af $\theta$}

\begin{minipage}[c]{.48\linewidth}
En parameter:

\includegraphics[width=\linewidth]{performance_one_par.md}

Model:\quad $\displaystyle\blue \frac{Y_m(s)}{U(s)} =
\frac{1}{1+s\tau}$
\end{minipage}\hspace*{\fill}
\begin{minipage}[c]{.48\linewidth}
To parametre:

\includegraphics[width=\linewidth]{performance_two_par.md}

Model:\quad $\displaystyle\blue \frac{Y_m(s)}{U(s)} =
\frac{K}{1+s\tau}$

%\put(175,150){\includegraphics[width=40pt]{tavle.md}}
\end{minipage}

\end{slide}

%---------------------------------------------------------------------- SLIDE -
\overlays{2}{
\begin{slide}[Replace]{Minimum af en funktion}

Betingelser for {\blue minimum i $\theta = \theta_0$} af en fkt. af flere
variable
$$P(\theta) = \frac{1}{2N} \sum_{k=1}^N (y(kT) - y_m(kT,\theta))^2$$

er, at {\blue gradient vektoren er nul}:\quad
$\displaystyle G(\theta_0) = \frac{\partial P(\theta)}{\partial
  \theta}\Big|_{\theta=\theta_0} = 0$

\bigskip
og at {\blue Hessian matricen}:\quad
$\displaystyle H(\theta_0) = \frac{\partial^2 P(\theta)}{\partial
  \theta \partial \theta^\top}\Big|_{\theta=\theta_0}$

\bigskip
er {\blue positiv definit}, dvs. $v^\top Hv > 0$ for alle $v \neq 0$.

\bigskip
\fromSlide{2}{
{\red Problem:} $G(\theta_0) = 0$ har generelt ingen eksplicit
l�sning!
}
\end{slide}
}
%---------------------------------------------------------------------- SLIDE -
\overlays{3}{
\begin{slide}[Replace]{Numeriske metoder til at finde minimum}

\begin{itemstep}
\item \makebox[.6\linewidth][l]{Steepest descent}\raisebox{-.75cm}{\includegraphics[height=1.5cm]{tavle.md}}
\item \makebox[.6\linewidth][l]{Newtons metode}\raisebox{-.75cm}{\includegraphics[height=1.5cm]{tavle.md}}
  + DEMO 
\item \makebox[.6\linewidth][l]{Gauss-Newton metoden}\raisebox{-.75cm}{\includegraphics[height=1.5cm]{tavle.md}}
\end{itemstep}

\end{slide}
}
%---------------------------------------------------------------------- SLIDE -
\overlays{4}{
\begin{slide}[Replace]{Direkte estimering af fysiske parametre}

\vspace*{-2mm}
\begin{itemstep}
\item Bestem model output (simulation): $\displaystyle\blue y_m(k) = F(u_n,\theta)$
\item Bestem model gradienten $\psi$ ved numerisk differentiation:

\hspace*{\fill}$\displaystyle\blue \psi_j(k,\theta) = \frac{y_m(k,\theta_j+\Delta\theta_j) -
  y_m(k,\theta_j)}{\Delta\theta_j}$\hspace*{\fill}

\item Bestem gradienten $G$ og Hessian matricen $H$ fra $\psi$:

{\small$\displaystyle\blue G(\theta) = -\frac{1}{N}\!\sum_{k=1}^N
\varepsilon(k,\theta)\psi(k,\theta)$,\hspace*{\fill}
$\displaystyle\blue \widetilde{H}(\theta) = \frac{1}{N} \!\sum_{k=1}^N
\psi(k,\theta)\psi^\top(k,\theta)$}

\item Bestem de parameter v�rdier der minimerer performance funktionen
  $P$ vha. Gauss-Newton metoden 

\hspace*{\fill}$\displaystyle\blue \theta_{i+1} = \theta_i -
\widetilde{H}^{-1}(\theta_i)G(\theta_i)$\hspace*{\fill}

\end{itemstep}

\end{slide}
}
%---------------------------------------------------------------------- SLIDE -

\begin{slide}[Replace]{N�ste Forel�sning}

\bigskip\bigskip
N�ste gang ser vi p�:

\begin{itemize}
\item Modeller og modellering: koncepter
\item Model beskrivelse
\item Diskritiseringsmetoder
\item Simulering af line�re og uline�re dynamiske systemer i Matlab
\end{itemize}

\end{slide}

%---------------------------------------------------------------------- SLIDE -

\end{document}


%%% Local Variables: 
%%% mode: latex
%%% TeX-master: master
%%% End: 
